\section{Definitions}
\label{sec:distr-defs}

We collect here a quick summary of definitions and properties of distributions.
These are commonly used in QFT -- often without paying too much attention to
them. 

\begin{Def}
    A {\em distribution} is a continuous linear functional on a space of 
    test functions. 
\end{Def}
The space of test functions may vary. We will mostly focus on {\em tempered}
distributions, \ie distributions that act on
$\mathcal{S}\left(\mathbb{R}^n\right)$, the space of complex-vaued, infinitely
differentiable functions, which, together with their derivatives, approach zero
at infinity, faster than any inverse power of the Euclidean distance. In order
to make this statement mathematically precise, we need to introduce some
notation. We denote by $k$ a generic $n$-tuple of integers, 
\begin{equation}
    \label{eq:MultiIndexDef}\
    k = \left\{ k_1, \ldots k_n\right\}\, ,
\end{equation}
then 
\begin{align}
    \label{eq:modk-def}
    |k| &= k_1 + \ldots + k_n\, , \\
    \label{eq:x-power-k}
    x^k &= x_1^{k_1} \ldots x_n^{k_n}\, , \\
    \label{eq:factk-def}
    k! &= k_1! \ldots k_n!\, , \\
    \label{eq:dx-def}
    dx &= \prod_{i=1, \ldots, n} dx_i\, , \\
    \label{eq:derivs}
    D^k &= \frac{\partial^{|k|}}{\partial x_1^{k_1} \ldots \partial x_n^{k_n}}\, .
\end{align}
For each pair of integers, $(r,s)$, a norm can be defined for functions in
$\mathbb{R}^n$,
\begin{Def}
    \[
    ||f||_{r,s} = \sum_{|k|\leq r} \sum_{|l|\leq s} \, 
    \sup_{x\in \Rn} \left| x^k D^l f(x)\right| \, .
    \]
\end{Def}
The set of tempered functions is the set of functions with finite norm, 
\[
    f \in \mathcal{S}(\Rn) \quad 
    \Longleftrightarrow \quad
    ||f||_{r,s} < \infty\, , \quad \forall r,s \in \mathbb{N}\, .    
\]
We can now introduce the concept of convergence in $\mathcal{S}$.
\begin{Def}
    The series $\{f_m\}$ converges to $f$, 
    \[
        \lim_{m\to\infty} f_m = f\, , \quad \mathrm{if} \ 
        \lim_{m\to\infty} ||f_m -f||_{r,s} = 0 \, .
    \] 
\end{Def}
Finally, we can define the notion of continuity for functionals acting on
$\mathcal{S}$. A {\em tempered distribution} $T$ is continuous in $\mathcal{S}$,
\ie
\[
    \lim_{m\to\infty} ||f_m - f||_{r,s} \quad
    \Longrightarrow \quad
    \lim_{m\to\infty} |Tf_m - Tf| = 0\, .
\]
Note that a necessary and sufficient condition for continuity is 
\[
    |Tf| \leq C ||f||_{r,s}\, , 
    \quad \mathrm{for\ some\ values\ of}\ r,s\, .    
\]
A very useful result states that every tempered distribution can be written as
\begin{align}
    \label{eq:TempConv}
    Tf 
        &= \sum_{|k|\leq s} \int dx\, 
            F_k(x_1 \ldots x_n) D^k f(x_1 \ldots x_n) \\
        &= \int dx\, T(x) f(x)\, ,
\end{align}
where $k$ is a multi-index as defined in Eq.~\eqref{eq:MultiIndexDef}, and the
coefficients $F_k$ are continuous functions that satisfy
\begin{equation}
    \label{eq:FkBound}
    \left|F_k(x)\right| \leq C_k \left(1 + |x|^j\right)\, ,
\end{equation}
for some $j$ that depends on $k$. We integrate by parts (when possible) and
write
\begin{align}
    \label{eq:TempKernel}
    T(x) = \sum_{|k|\leq s} (-)^{|k|} D^k F_k(x)\, .
\end{align}
The set of tempered distributions is denoted $\mathcal{S}'$. At times we will
encounter distributions that do not belong to $\mathcal{S}'$, but to the larger
set $\mathcal{D}'$, of linear continuous functionals acting on $\mathcal{D}$,
the space of infinitely differentiable functions with compact support in $\Rn$.
Clearly $\mathcal{D}\subset\mathcal{S}$ and therefore $\mathcal{S}' \subset
\mathcal{D}'$. Examples of elements of $\mathcal{D}'$ that do not belong to
$\mathcal{S}'$ are exponentially growing functions, or 
\[
    T(x) = \sum_{m=0}^\infty \delta^{(m)}(x-m)\, .   
\]

\section{Miscellaneous Properties}
\label{sec:MiscProp}

Let us consider invertible inhomogeneous transformations in $\Rn$
\begin{equation}
    \label{eq:RnInhomogeneous}
    \{a, L\}:\, x \mapsto L x + a\, ,
\end{equation}
for any function $f\in\mathcal{S}$, we define the {\em shifted} function
\begin{equation}
    \label{eq:ShiftedFun}
    \left(\{a,L\} f\right)(x) = 
        f\left(L^{-1}(x-a)\right)\, .
\end{equation}
The shifted distribution in $\mathcal{S}'$ is defined by
\begin{equation}
    \label{eq:ShiftedDistr}
    T_{\{a,L\}} f = \left|\det L\right|^{-1} 
        T\left(\{a,L\}f\right)\, .
\end{equation}
If $T$ is a function
\begin{equation}
    \label{eq:ShiftedDistrFun}
    T_{\{a,L\}}(x) = T(Lx+a)\, .
\end{equation}

The derivatives of a distribution are defined by the relation
\begin{equation}
    \label{eq:DistrDeriv}
    D^p T(f) = (-)^{|p|}\, T\left(D^pf\right)\, .
\end{equation}
Note that using the definition in Eq.~\eqref{eq:ShiftedDistr}, the derivative of
a distribution can be equivalently defined as
\begin{equation}
    \label{eq:EquivDerivDef}
    \frac{\partial}{\partial x_j} T (f) =
    \lim _{a_j\to 0} \frac{1}{a_j} \left(T_{\{a_j\}} - T\right)(f)\, .
\end{equation}
We do not want to write a book on distributions here, so we will not dwell on
proving nice identities. Instead we refer the reader to the book by Streater \&
Wightman for the details.  

Multiplication by a function, tensor product of distributions, and convolution
with a function can be defined in the {\em obvious} way. The only subtlety to
keep in mind is that when we multiply a tempered distribution by a generic,
infinitely differentiable function $g$, the result may not be a tempered
distribution. In order to guarantee that we recover a tempered distribution, we
need to require $g$ and all its derivatives are bounded by polynomials. 

\paragraph{Nuclear Theorem}

It is necessary at times to consider multilinear functionals that are separately
continuous in all their arguments. An obvious way to construct such
distributions is to consider a distribution $G$ in all the variables together,
and specialize it to test functions that are products $f_1(x_1) \ldots
f_m(x_m)$. The nuclear theorem found by Schwartz guarantees that these are the
only possible distributions. 
\begin{Thm}
    Let $T$ be a multilinear functional of arguments 
    $f_1, \ldots, f_m \in \mathcal{S} (\in \mathcal{D})$, which is continuous in 
    each of its arguments while the others are fixed, then there exists a unique 
    distribution $G \in \mathcal{S}' (\in \mathcal{D}')$ such that
    \[
        T(f_1, \ldots, f_m) = G(f_1 \ldots f_m)\, .    
    \]
\end{Thm}

\section{Fourier Transforms}
\label{sec:FTDistr}

Let us recall briefly the conventions used here for the Fourier transform. In
Euclidean space, we use
\begin{align}
    P_k = -i \partial_k 
    \quad \Longrightarrow \quad 
    \braket{x}{p} = e^{i p\cdot x}\, .
\end{align}
The Fourier transform is 
\begin{align}
    \left(\mathcal{F}f\right)(p) 
        &= \braket{p}{f} = \int d^Dx\, \braket{p}{x} \braket{x}{f} \\
        &= \int d^Dx\, e^{-i p\cdot x} f(x)\, ,
\end{align}
and similarly for the inverse transform
\begin{align}
    \left(\bar{\mathcal{F}} f\right)(x) = 
    \int \frac{d^Dp}{(2\pi)^D}\, e^{i p\cdot x} 
    f(p)\, .
\end{align}
With these conventions
\begin{align}
    \mathcal{F} \bar{\mathcal{F}} =
    \bar{\mathcal{F}} \mathcal{F} = 
    1\, .
\end{align}
Using these definitions we have
\begin{equation}
    \label{eq:Fdkf}
    \mathcal{F}\left(D^kf\right)(p) = 
        \left(ip\right)^k \left(\mathcal{F}f\right)(p)\, .
\end{equation}

\begin{Def}
    The Fourier transform of a distribution $T$ is defined by imposing
    \begin{equation}
        \label{eq:FTDistrDef}
        \left(\mathcal{F}T\right) (f) = 
        T\left(\mathcal{F}f\right)\, .
    \end{equation}
\end{Def}

\begin{Ex}
    Show that with the definitions above
    \begin{align}
        \mathcal{F}[\delta(x)] 
            &= 1\, ,    \\
        \mathcal{F}[e^{ikx}] 
            &= (2\pi) \delta(p-k)\, \quad \forall k \in \mathbb{R}\, .
    \end{align}
    where $\delta(x)$ is Dirac's delta function (which is actually a
    distribution). Check that you can generalize these expressions to the case
    of $x,p \in \Rn$.
\end{Ex}

\section{Laplace Transform}
\label{sec:LaplaceTransf}

\begin{Def}
    The Laplace transform of a function $f\in \mathcal{S}(\mathbb{R})$ is 
    defined as
    \begin{align}
        \label{eq:LaplaceTransfDef}
        g(z) = \int_0^\infty dk\, e^{i k (x+iy)} f(k)\, , \quad z=x+iy\, .
    \end{align}
\end{Def}
The function $g(z)$ is holomorphic in the plane $y>0$. This definition can be
generalized to the case of multiple variables, where the integral over the
positive $k$ axis is replaced by an integral in a {\em convex cone}. We admit
distributions as boundary values of these Laplace transforms. The generalization
of the upper half-plane is a so-called {\em tube}, where the imaginary parts of
the complex variables are constrained to be in a cone. In the multi-dimensional
case, we can extend the definition of holomorphic function as follows. 

\begin{Def}
    A function $f$, defined in a neighbourhood of $w \in \Cn$, is {\em
    holomorphic} (or analytic) at the point $w$ if there exists a series
    \begin{align}
        \label{eq:HoloSeriesDef}
        \sum_{k_1, \ldots, k_n=0}^\infty &a_{k_1 \ldots k_n} (z_1-w_1)^{k_1} \ldots
            (z_n - w_n)^{k_n} \\
        & \sum_k a_k (z-w)^k\, ,
    \end{align}
    which converges for $z$ in a neighbourhood of $w$, and is equal to $f(w)$
    when $z=w$. 
\end{Def}
Note that the second line is a compact rewriting of the first one using the
multi-index notation. If the series converges at $z$, then it converges
absolutely and uniformly in the polydisc
\begin{align}
    \left| \zeta_j - w_j \right| \leq |z_j - w_j| - \epsilon\, .
\end{align}
The coefficients are obtained by differentiating the function at $w$, 
\begin{align}
    \label{eq:CoeffsFromDerivs}
    a_{k_1\ldots k_n} = 
        \left. \frac{1}{k!} D^{k}f(z_1, \ldots, z_n) \right|_{w_1=z_1 \ldots w_n=z_n}\, .
\end{align}
Cauchy's formula yields
\begin{align}
    f(z_1, \ldots, z_n) = \frac{1}{(2\pi i)^n} \int_{R_1} \ldots \int_{R_n}
        d\zeta_1 \ldots d\zeta_n\,  
        \frac{f(\zeta_1, \ldots, \zeta_n)}{(\zeta_1 - z_1) \ldots (\zeta_n - z_n)}\, , 
\end{align}
inside a polydisc of uniform convergence. Holomorphic functions of several
variables can be extended by analytical continuation. The coefficients of the
power series that defines a holomorphic function can be obtained by computing
the derivatives along the real axis, \ie by varying only the real part of the
complex arguments. 

A holomorphic function $F$ defined in an open set $\mathcal{O}$ is a
distribution in $\mathcal{D}(\mathcal{O})'$. If $T$ is a distribution in
$\mathcal{D}_p'$, it may happen that $e^{-\eta \cdot p} T$ is a distribution in
$\mathcal{S}_p'$. Then we can define the Laplace transform of $T$ as the
distribution
\begin{align}
    \label{eq:LaplaceDistrDef}
    \mathcal{L}(T) = \mathcal{F}\left(e^{\eta\cdot p} T\right)\, .
\end{align}
For a function $T$ we get
\begin{align}
    \label{eq:LaplaceDistrFunc}
    \mathcal{L}(T)(\xi,\eta) = 
        \int \frac{d^Dp}{(2\pi)^D}\, e^{-i p (\xi - i \eta)} T(p)\, .
\end{align}
Here there are no requirements on the support of $T$; this is called a two-sided
Laplace transform. The one-dimensional definition in
Eq.~\eqref{eq:LaplaceTransfDef} is a special case of this more general
definition. 

\subsection{Properties}
\label{sec:LaplaceProperties}

We list here some useful properties, without getting into the details of the
proofs.

\begin{Thm}
    Let $T \in \mathcal{D}_p'$, the set of all $\eta$ such that 
    $e^{-p\cdot \eta} T \in \mathcal{S}_p'$ is {\em convex}.
\end{Thm}
The proof is very nice, but we do not have time for it right now. See Thm. 2.5
in Streater \& Wightman for the details. 

\begin{Thm}
    Let $\Gamma$ be a convex open set in $\Rn$, $T \in \mathcal{D}_p'$ such that
    \[
        \forall \eta \in \Gamma\, , e^{-\eta\cdot p} T \in \mathcal{S}_p'\, ,   
    \]
    then $\mathcal{L}(T)$ is a holomorphic function of $\xi-i\eta$ in the tube
    $\Rn - i\Gamma$. $\mathcal{L}(T)$ satisfies
    \begin{align}
        \label{eq:PolyBound}
        |\mathcal{L}(T)(\xi-i\eta)| \leq \left|P_k(\xi)\right|\, ,           
    \end{align}
    where $P_k$ is a polynomial, and $\eta$ varies over any compact 
    subset $K \subset\Gamma$.

    Conversely, every function holomorphic in the tube, and satisfying the bound
    in Eq.~\eqref{eq:PolyBound} is the Laplace transform of a uniquely
    determined $T \in \mathcal{D}_p'$, such that $e^{-\eta\cdot p} T \in
    \mathcal{S}_p'$ for all $\eta\in\Gamma$.
\end{Thm}

\begin{Thm}
    \label{thm:TranslationTube}
    Let $T\in\mathcal{D}_p'$ and $\Gamma\in\Rn$, convex, such that 
    \[
        \forall\eta\in\Gamma\, , e^{-p\cdot\eta} T \in \mathcal{S}_p'\, .    
    \]
    If $T$ has its support in a half-space such that $p\cdot a > A$, then
    $\Gamma$ contains all the points of the form $\eta + t a$, with $t\geq 0$.
\end{Thm}

The `plus' lightcone, $V_+$, is defined as the set of $D$-momenta $p$ such that
\begin{align}
    \label{eq:PlusLightConeDef}
    p^2 = (p^0)^2 - \mathbf{p}^2 > 0\, , \quad p^0 > 0\, .
\end{align}
Its closure is denoted $\bar{V}_+$. In QFT we find distributions $T$ that are
functions of $n$ $D$-momenta $p_1,\dots, p_n$. These distributions vanish if
some momentum $p_i$ is outside $\bar{V}_+$, and are also tempered. Using
Theorem~\ref{thm:TranslationTube}, we deduce that $\mathcal{L}(T)(\xi-ia)$ is
analytic in the {\em tube}:
\begin{align}
    \label{eq:ForwardTube}
    \mathcal{T}_n=\Rn - i\Gamma\, ,
\end{align}
where
\[
    \Gamma = \left\{
        (a_1, \ldots a_n)\, , \, a_j \in V_+\, , \, j=1, \ldots n
    \right\}\, .    
\]
Note that here we use the fact that if $p,q \in V_+$, then $p\cdot q > 0$. 
For functions that have support in the cone $\mathcal{T}_n$, we have the 
following property. 

\begin{Thm}
    Let $T \in \mathcal{D}_p'$ and $e^{-p\cdot\eta} T \in\mathcal{S}_p'$ 
    for $\eta\in\Gamma$, where $\Gamma$ is the cone $\eta_j\in V_+$ for 
    $j=1, \ldots, n$. Let also 
    \begin{equation}
        p \in \mathrm{supp}\ T 
        \quad \Longrightarrow \quad
        p_j \in V_+\, , j=1, \ldots, n\, .    
    \end{equation}
    Then for each $\eta\in\Gamma$, there is a polynomial $P_\eta$ 
    such that
    \begin{align}
        \label{eq:InequalityAbove}
        |\mathcal{L}(T)\left(\xi - i (\eta+a)\right)|
        \leq |P_\eta(\xi-ia)|\, ,
    \end{align}
    for all $\xi$ and all $a\in\Gamma$.

    Conversely, if $F$ is holomorphic in $\mathcal{T}_n$ and 
    satisfying Eq.~\eqref{eq:InequalityAbove} for each 
    $\eta\in\Gamma$ and some polynomial $P_\eta$, then $F$ is the 
    Laplace transform of a distribution with support in $\Gamma$.
\end{Thm}

The existence of a boundary value of Laplace transforms is dictated
by the following theorem. 
\begin{Thm}
    If $T\in\mathcal{S}_p'$ and $\mathcal{L}(T)$ exists for all $\eta\in\Gamma$, 
    then
    \begin{align}
        \label{eq:BoundaryValueThm}
        \lim_{\eta\to 0} \int d\xi\, \mathcal{L}(T)(\xi-i\eta) f(\xi) 
        = \mathcal{F}(T)(f)\, ,
    \end{align}
    \ie $\mathcal{L}(T)$ converges in $\mathcal{S}_p'$ to $\mathcal{F}(T)$ 
    as $\eta\to 0$ inside a cone $\Gamma$.

    Conversely if $\mathcal{L}(T)$ converges in $\mathcal{S}_\xi'$ as $\eta\to 0$
    in a cone $\Gamma$, then $T$ is the Laplace transform of a tempered distribution.
\end{Thm}

\section{Extended Tubes}
\label{sec:ExtTubes}

The {\em extended tube}, $\mathcal{T}_n'$, is the union of the open sets
obtained from $\mathcal{T}_n$ by applying proper complex Lorentz
transformations:
\begin{align}
    &\zeta_1, \ldots, \zeta_n \in \mathcal{T}_n' 
    \quad \Longleftrightarrow \quad
    \zeta_1, \ldots, \zeta_n = \Lambda w_1, \ldots, \Lambda w_n\, , \\
    &\quad w_1, \ldots, w_n \in \mathcal{T}_n\, ,  \quad
    \Lambda \in L_+(\mathbb{C})\, . \nonumber
\end{align}
The holomorphic functions that we find in QFT transform according to
some representation $S(A)$ of $\mathrm{SL}(2,\mathbb{C})$:
\begin{align}
    \label{eq:TransfPropFunc}
    f_\alpha(\Lambda(A)\zeta_1, \ldots, \Lambda(A)\zeta_n) =
    S(A)_{\alpha\beta} f_\beta(\zeta_1, \ldots, \zeta_n)\,.
\end{align}
The main result about extended tubes in Streater \& Wightman is the following
theorem.
\begin{Thm}
    If $f_\alpha(\zeta_1, \ldots, \zeta_n)$ transforms according to 
    Eq.~\eqref{eq:TransfPropFunc} and is holomorphic in the tube 
    $\eta_j \in V_+$, where $\zeta_j=\xi_j-i\eta_j$, $j=1, \ldots, n$, 
    then $f_\alpha$ possesses a single-valued analytical continuation
    into the extended tube $\mathcal{T}_n'$.
\end{Thm}

\paragraph{Jost points}

By definition the tube $\mathcal{T}_n$ does not contain real points, 
since $z \in \mathcal{T}_n$ implies $\mathrm{Im}\ z \in V_+$, and 
therefore $\mathrm{Im}\ z \neq 0$.The extended tube contains real points, 
called {\em Jost points}. 

\begin{Ex}
    For the case of one vector, show that a real point 
    $\zeta \in \mathcal{T}_1'$ if and only if $\zeta^2 < 0$.
\end{Ex}

For the general case, we have a theorem by Jost. 
\begin{Thm}
    A real point $\zeta_1,\ldots,\zeta_n$ is in the extended tube
    $\mathcal{T}_n'$ if and only if all vectors of the form 
    \[
     \sum_{j=1}^n \lambda_j \zeta_j\, , \quad \lambda_j\geq 0\, , \sum_j \lambda_j >0\, ,
    \]
    are space-like.
\end{Thm}

\section{The Edge of the Wedge Theorem}
\label{eq:EdgeOfTheWedge}

For one complex variable there is a proof of the edge of the wedge 
theorem that goes back to Painlev\'e in 1888. 

\begin{Thm}
    Let $F_1$ be a holomorphic function in an open set $D_1$ in the upper half
    plane, with an interval $a<x<b$ as part of its boundary. Let $F_2$ be
    holomorphic in an open set $D_2$ in the lower half-plane, with the interval
    $a<x<b$ as part of its boundary. Suppose 
    \begin{align}
        F_1(x) &= \lim_{y\to 0^+} F_1(x+iy) \nonumber \\
        F_2(x) &= \lim_{y\to 0^+} F_2(x-iy) \nonumber 
    \end{align}
    exist uniformly in $a<x<b$, are continuous and satisfy
    \begin{align}
        F_1(x) = F_2(x)\, , \quad \forall x\in (a,b)\, .
    \end{align}
    Then $F_1$ and $F_2$ are the same holomorphic function on 
    $a<x<b$.
\end{Thm}
The proof is reported in Streater \& Wightman. Add it to the notes at a later
stage. 

\paragraph{Schwarz reflection principle} 

This is a corollary of the theorem above. It can be stated as follows. 

If $F_1$ is holomorphic in $D_1$ and converges uniformly to boundary values for
$a<x<b$ which define a real continuous function, then $F_1$ is holomorphic in
$\bar{D}_1$, and
\begin{equation}
    \label{eq:SchwarzAnalytical}
    F_1(z) = F_1(z^*)^*
\end{equation}
defines its analytical continuation. 

For applications in QFT we need a generalization of the edge of the wedge
theorem to the case of several complex variables and to distributions. We are
not going to report the proofs, as they can be found in Streater \& Wightman. We
only state two theorems, which summarise the properties that we encounter in
QFT. 

\begin{Thm}
    \label{thm:EdgeWedgeMultiDim}
    Let $\mathcal{O}$ be open in $\Cn$, containing an open set $E$ of $\Rn$. 
    Let $\mathcal{C}$ be an open convex cone of $\Rn$. Suppose $F_1$ is 
    holomorphic in
    \begin{equation}
        D_1 = \left(\Rn + i \mathcal{C}\right) \cap \mathcal{O} 
    \end{equation} 
    and $F_2$ in 
    \begin{equation}
        D_2 = \left(\Rn - i \mathcal{C}\right) \cap \mathcal{O} \, .
    \end{equation} 
    Suppose the limits for $x\in E$
    \begin{equation}
        \label{eq:UpperEdgeFun}
        \lim_{y\to 0,\; y\in \mathcal{C}} F_1(x+iy) = F_1(x) 
    \end{equation}
    and
    \begin{equation}
        \label{eq:LowerEdgeFun}
        \lim_{y\to 0,\; y\in \mathcal{C}} F_2(x-iy) = F_2(x) 
    \end{equation}
    exist and are continuous and coincide on $E$, with the limit
    being uniform on $E$. 

    Then there exists a complex neighbourhood $N$ of $E$ and a 
    holomorphic function $G$ which coincides with $F_1$ in $D_1$ 
    and $F_2$ in $D_2$. 
\end{Thm}

A similar theorem is valid even in the case where the boundary 
values do not define a continuous function. 

\begin{Thm}
    \label{thm:EdgeWedgeDistr}
    Let us replace hypotheses~\eqref{eq:UpperEdgeFun} 
    and~\eqref{eq:LowerEdgeFun} with the condition that for every 
    test function of compact support in $E$
    \begin{equation}
        \label{eq:UpperEdgeDistr}
        \lim_{y\to 0,\; y\in \mathcal{C}} 
        \int dx\, F_1(x+iy) f(x) = T(f)
    \end{equation}
    and 
    \begin{equation}
        \label{eq:LowerEdgeDistr}
        \lim_{y\to 0,\; y\in \mathcal{C}} 
        \int dx\, F_2(x-iy) f(x) = T(f)\, ,
    \end{equation}
    where $T$ is a distribution in $\mathcal{D}(E)'$.
\end{Thm}
