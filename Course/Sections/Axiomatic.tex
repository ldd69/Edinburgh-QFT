\newcommand{\Hin}{\mathcal{H}_{\mathrm{in}}}

\section{Time Evolution Pictures in Quantum Mechanics}
\label{sec:reps-quant}

In order to set the notation, we briefly summarise the different representations
used to describe the time evolution of quantum systems. The state of the system
at $t=0$ is described by a vector $\ket{\psi} \in \mathcal{H}$, where
$\mathcal{H}$ is the Hilbert space of physical states.

\paragraph{Schr\"odinger Picture}

In the Schr\"odinger representation the time evolution is encoded in the time
dependence of the state vector, $\ket{\psi_S(t)}$, which evolves according to
the Schr\"odinger equation
\begin{equation}
    \label{eq:SchrodEq}
    i \partial_t \ket{\psi_S(t)} = H \ket{\psi_S(t)}\, ,
\end{equation}
where $H$ is the Hamiltonian operator for the system under study. Operators
associated to observables, $O_S$, are time independent (unless the observable
itself has an explicit dependence on time). The expectation value of the
observable at time $t$ is 
\begin{equation}
    \label{eq:SchrodExp}
    O(t) = \langle \psi_S(t) | O_S | \psi_S(t) \rangle\, .
\end{equation}

\paragraph{Heisenberg Picture}

In the Heisenberg representation, the state vector does not evolve in time. At
all times $t$, $\ket{\psi_H(t)}=\ket{\psi}$. We will therefore drop the suffix
$H$ when we refer to states in the Heisenberg representation. Time evolution is
encoded in the time dependence of the operators $O_H(t)$. Clearly the
expectation value at time $t$ should not depend on whether we work in the
Schr\"odinger or Heisenberg representation:
\begin{equation}
    \label{eq:HeisenExp}
    O(t) = \langle \psi | O_H(t) | \psi \rangle\, .
\end{equation}
Comparing Eqs.~\eqref{eq:SchrodExp} and~\eqref{eq:HeisenExp} we deduce that
\begin{equation}
    \label{eq:HeisenOp}
    O_H(t) = e^{iHt} O_S e^{-iHt}\, ,
\end{equation}
and therefore 
\begin{equation}
    \label{eq:HeisenEvol}
    i \frac{d}{dt} O_H(t) = - \left[H, O_H(t)\right]\, ,
\end{equation}
where $H$ is the Hamiltonian again.~\footnote{Clearly, the Hamiltonian operator
in the Heisenberg representation is time-independent. Do not confuse the
Hamiltonian $H$ and the suffix $H$, used to denote the Heisenberg
representation.}

\paragraph{Interaction Picture}

The interaction picture interpolates between the previous two. In this case the
Hamiltonian is divided into a {\it free} Hamiltonian $H_0$ and an interaction
term $V$. The time dependent state vector in the interaction picture is defined
by 'subtracting' the free evolution from the Schr\"odinger state vector at time
$t$:
\begin{equation}
    \label{eq:IntPictEvol}
    \ket{\psi_{\mathrm{int}}} = e^{i H_0 t} \ket{\psi_S(t)}\, .
\end{equation}
The operators are also time-dependent:
\begin{equation}
    \label{eq:IntPictOps}
    O_{\mathrm{int}}(t) = e^{i H_0 t} O_S e^{-i H_0 t}\, .
\end{equation}
In this representation the free Hamiltonian is time independent,
\begin{equation}
    \label{eq:H0IntPict}
    H_{0,\mathrm{int}}(t) = H_0\, ,
\end{equation}
and the time evolution of the states is described by a first-order differential
equation similar to Schr\"odinger's equation, 
\begin{equation}
    \label{eq:IntPictEvolEq}
    i \partial_t \ket{\psi_{\mathrm{int}}(t)} = V_{\mathrm{int}}(t) 
    \ket{\psi_{\mathrm{int}}(t)}\, .
\end{equation}

Finally, note that fields in QFT are treated as operator-valued distributions
and therefore
\begin{equation}
    \label{eq:HvIntFields}
    \phi_H(t,\mathbf{x}) = U(t)^\dagger \phi_{\mathrm{int}}(t,\mathbf{x}) 
    U(t)\, ,
\end{equation}
where
\begin{equation}
    \label{eq:Uoperator}
    U(t) = e^{iH_0t} e^{-iHt}\, .
\end{equation}

\paragraph{Conventions for Generators of Translations}

Using a $D$-dimensional covariant notation, and a Minkovski metric 
\begin{equation}
    \label{eq:MostlyMinus}
    \eta_{\mu\nu} = \mathrm{diag}\left\{1, -1, -1, \ldots \right\}\, ,
\end{equation}
we can write the momentum operator in position space as
\begin{equation}
    \label{eq:MomOp}
    P^\mu = i \partial^\mu\, ,
\end{equation}
which reduces to the usual expressions for the Hamiltonian $P^0$ and the spatial
components of the momentum $P^k$. The momentum being the generator of
translations, the wave function of a system in the Schr\"odinger picture obeys
\begin{equation}
    \label{eq:PsiTranslation}
    \psi\left(x+a\right) = e^{-i P\cdot a} \psi\left(x\right)\, ,
\end{equation}
while for the field operators in the Heisenberg representation
\begin{equation}
    \label{eq:PhiTranslation}
    \phi_H(x+a) = e^{i P\cdot a} \phi_H(x) e^{-i P\cdot a}\, .
\end{equation}

\section{Asymptotic States}
\label{sec:AsymptStates}

'In' and 'Out' states are states of the interacting theory in the Heisenberg
representation, which are characterized by the behaviour of the system in the
far past and the far future respectively, where the particles can be considered
to be well separated. The separation between particles implies that these states
should be in a one-to-one correspondence with free-particle states. Let us focus
here on in-states, similar results hold for the out-states. The Hilbert space of
in-states is denoted $\Hin$. A basis of $\Hin$ is made of simultaneous momentum
eigenstates of $N$ particles, where $N$ spans the set of integer numbers
\begin{equation}
    \label{eq:HinBasis}
    \left\{ \ket{k_1 \ldots k_N}; \mathrm{in}, N \in \mathbb{N} \right\}\, .
\end{equation}
Clearly the individual momenta are not conserved quantities, while translation
invariance guarantees that the total momentum $P$ is conserved
\begin{equation}
    \label{eq:TotMomConserv}
    P^\mu \ket{k_1 \ldots k_N; \mathrm{in}} = 
    \left(\sum_{m=1}^N k^\mu_m\right) 
    \ket{k_1 \ldots k_N; \mathrm{in}}\, .
\end{equation}
From the definitions summarised in Sec.~\ref{sec:reps-quant}, we see that states
in the Schr\"odinger and Heisenberg pictures coincide at $t=0$. We can express
the fact that in-states behave like free-particle states as $T \to -\infty$. Let
$\ket{a; 0}$ be a free-particle state, we require
\begin{equation}
    \label{eq:InAsympOne}
    \lim_{T\to-\infty} e^{-i H_0 T} \ket{a;0} =
    \lim_{T\to-\infty} e^{-i H T} \ket{a; \mathrm{in}}\, ,
\end{equation}
which we rewrite as~\footnote{There is potential for sloppiness here, but I
don't think it does matter.}
\begin{align}
    \label{eq:InAsympTwo}
    \ket{a; \mathrm{in}} &= \lim_{T\to-\infty} e^{i H T} e^{-i H_0 T} \ket{a;0}
    \\
    \label{eq:MollerOne}
    &= \Omega^+ \ket{a; 0}\, .
\end{align}
Note that 
\begin{equation}
    \label{eq:MollerComment}
    \Omega^+ = \lim_{T\to-\infty} U(T)^\dagger.
\end{equation}
Similarly for out-states,
\begin{align}
    \label{eq:OutAsympTwo}
    \ket{a; \mathrm{out}} &= \lim_{T\to\infty} e^{i H T} e^{-i H_0 T} \ket{a;0}
    \\
    \label{eq:MollerTwo}
    &= \Omega^- \ket{a; 0}\, .
\end{align}
Finally we recall the definition of the $S$ matrix: 
\begin{align}
    \label{eq:SMatOne}
    S_{ab} &= \langle a; \mathrm{out} | b; \mathrm{in} \rangle \\
    \label{eq:SMatTwo}
    &= \langle a; 0 | \left(\Omega^-\right)^\dagger
    \Omega^+ | b; 0\rangle\, .
\end{align}


