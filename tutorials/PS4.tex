\documentclass[12pt,a4paper]{article}
%
\setlength{\oddsidemargin}{  0mm}
\setlength{\topmargin}    { -12mm}
\setlength{\textheight}   { 246mm}
\setlength{\textwidth}    { 165mm}
\setlength{\parindent}    {  0   pt}  % not actually required but they

\setlength{\parskip}      {  6   pt}  % make paragraphs look less ugly

%

\usepackage{amsmath}
\usepackage{amssymb}
\pagestyle{empty}

\pagestyle{headings}

\textwidth=440pt
\hoffset=-0.6truein

\usepackage{amsmath}
\usepackage{amsfonts}
\usepackage{amssymb}
\usepackage{dsfont}
\usepackage{pifont}
%\usepackage{bbold}
\usepackage{graphicx}
\usepackage{epstopdf}
\usepackage{epsfig}
%\usepackage{bibunits}
%\usepackage{theorem}
\usepackage[framed]{ntheorem}
\usepackage{framed}
%\usepackage{showlabels}
\usepackage{makeidx}
\usepackage{simplewick}
\usepackage{tikz-feynman}

\tikzfeynmanset{compat=1.0.0}  

\newcommand{\tick}{\ding{52}}
\newcommand{\notick}{\ding{56}}
\newcommand{\D}{\displaystyle}
\renewcommand{\chaptername}{Lecture}

\def\bfx{{\mathbf x}}
\def\bfxp{{\mathbf x^\prime}}
\def\bfy{{\mathbf y}}
\def\bfyp{{\mathbf y^\prime}}
\def\bfp{{\mathbf p}}
\def\bfpp{{\mathbf p^\prime}}
\def\ddt{\frac{d}{dt}}
\def\ddtt{\frac{d^2}{dt^2}}
\def\ie{{\it i.e.}\ }
\def\eg{{\it e.g.}\ }
\def\viz{{\it viz.}\ }
\def\matF{\mathcal F}
\def\matE{\mathcal E}
\def\GL{\mathrm{GL}}
\def\kpsi{|\psi\rangle}
\def\kpsione{|\psi_1\rangle}
\def\kpsitwo{|\psi_2\rangle}
\def\kpsionep{|\psi_1^\prime\rangle}
\def\kpsitwop{|\psi_2^\prime\rangle}
\def\kpsii{|\psi_i\rangle}
\def\kpsin{|\psi_n\rangle}
\def\kpsip{|\psi^\prime\rangle}
\def\bpsi{\langle\psi |}
\def\bpsione{\langle\psi_1 |}
\def\bpsitwo{\langle\psi_2 |}
\def\bpsii{\langle\psi_i |}
\def\bpsip{\langle\psi^\prime |}
\def\kphi{|\phi\rangle}
\def\kphione{|\phi_1\rangle}
\def\kphitwo{|\phi_2\rangle}
\def\kphii{|\phi_i\rangle}
\def\kphip{|\phi^\prime\rangle}
\def\bphi{\langle\phi |}
\def\bphione{\langle\phi_1 |}
\def\bphitwo{\langle\phi_2 |}
\def\bphii{\langle\phi_i |}
\def\bphip{\langle\phi^\prime |}
\def\bchi{\langle\chi |}
\def\bchione{\langle\chi_1 |}
\def\bchitwo{\langle\chi_2 |}
\def\bchii{\langle\chi_i |}
\def\bchip{\langle\chi^\prime |}
\def\kjm{|j,m\rangle}
\def\tr{\mathrm{Tr}}
\def\id{\mathds{1}}
{\theoremstyle{plain} \theorembodyfont{\rmfamily} \newframedtheorem{Ex}{Exercise}[section]}
{\theoremstyle{plain} \theorembodyfont{\rmfamily} \newtheorem{Def}{Definition}[section]}
{\theoremstyle{plain} \theorembodyfont{\rmfamily} \newtheorem{Thm}{Theorem}[section]}

\newcommand{\clearemptydoublepage}{\newpage{\pagestyle{empty}\cleardoublepage}}
\newcommand{\HRule}{\rule{\linewidth}{0.5mm}}
\newcommand{\iu}{\underline{i}}
\newcommand{\ju}{\underline{j}}
\newcommand{\ku}{\underline{k}}
\newcommand{\ru}{\underline{r}}
\newcommand{\pu}{\underline{p}}
\newcommand{\Lu}{\underline{L}}
\newcommand{\Ju}{\underline{J}}
\newcommand{\lap}{\nabla^2}
\newcommand{\ad}{\hat{a}}
\newcommand{\ac}{\hat{a}^\dagger}
\newcommand{\re}{\mathrm{Re}}
\newcommand{\ket}[1]{| #1 \rangle}
\newcommand{\bra}[1]{\langle #1 |}
\newcommand{\braket}[2]{\langle #1 | #2 \rangle}
\newcommand{\pref}[1]{(\ref{#1})}
\newcommand{\Eqref}[1]{Eq.~(\ref{#1})}
\newcommand{\del}{\v{\nabla}}				% Underlined del



\begin{document}
\begin{center}
{\bf Quantum Field Theory}\\[\baselineskip]
\end{center}
{\bf Problem Sheet 4}

\begin{enumerate}
  \item {\it Feynman rules - 1}\\

    Use the Feynman rules in momentum space to compute
    $G^{(2,2)}_b$. Check that you get the same result by performing a
    Fourier transform of the result in position space.

    \bigskip

  \item {\it Scattering amplitude} \\
    
      Compute the amplitude for the scattering process
      \[
        p_1 p_2 \longrightarrow p_1' p_2'
      \]
      at order $g^2$ in the $\phi^3$ scalar theory. 


    \bigskip
    
  \item {\it LSZ reduction for 2 to 2 processes}\\
    
    A particle localised in momentum space near $\mathbf{k}_1$ is
    created in $D=4$ dimensions by the operator
    \begin{align}
      \label{eq:1}
      a_1^\dagger &= \int d^3k  f_1(\mathbf{k}) a^\dagger(\mathbf{k})\, ,
    \end{align}
    where $f$ is some function peaked at $\mathbf{k}_1$, and
    $a^\dagger(\mathbf{k})$ is the creation operator in the free
    theory. In the interacting theory, we shall assume that a
    time-dependent creation operator is defined as
    \begin{align}
      \label{eq:2}
      a^\dagger(\mathbf{k},t) = -i \int d^3x\, e^{-ik\cdot x}
      \overleftrightarrow{\partial_0} \phi(x)\, .
    \end{align}
    Show that
    \begin{align}
      \label{eq:3}
      a_1^\dagger(+\infty) - a_1^\dagger(-\infty) = 
      -i \int d^3k\, f_1(\mathbf{k}) \int d^4x\,
      e^{-ik\cdot x} (\partial^2+m^2) \phi(x).
    \end{align}
 
    The scattering amplitude for a $2\rightarrow 2$ process can be
    written as: 
    \begin{align}
      \label{eq:4}
      \langle k_1' k_2';\mathrm{out} | k_1 k_2; \mathrm{in} \rangle 
      &= \langle 0 | T\left(
        a_{1'}(+\infty) a_{2'}(+\infty) 
        a_1^\dagger(-\infty) a_2^\dagger(-\infty)
        \right) |0 \rangle\, .
    \end{align}
    Show that
    \begin{align}
      \label{eq:5}
      \langle k_1' k_2';\mathrm{out} | k_1 k_2; \mathrm{in} \rangle 
      =&  i^{2+2} 
         \int d^4x_1\, e^{-i k_1\cdot x_1} \left(\partial_1^2 + m^2\right)  
         \int d^4x_2\, e^{-i k_2\cdot x_2} \left(\partial_2^2 +
         m^2\right)  \nonumber \\
      & \times \int d^4x_1'\, e^{i k_1'\cdot x_1'} \left(\partial_{1'}^2 + m^2\right)  
         \int d^4x_2'\, e^{i k_2'\cdot x_2'} \left(\partial_{2'}^2 +
        m^2\right) \nonumber \\
      & \times \langle 0 | T\left(
        \phi(x_1) \phi(x_2) \phi(x_1') \phi(x_2')
        \right) |0 \rangle\, .        
    \end{align}

\item {\it Generalised LSZ}

  Generalise the LSZ reduction for arbitrary numbers of particles in
  the initial and final state. 

\end{enumerate}

\vfill
\hspace*{\fill}\tiny L Del Debbio, October 2018.
\end{document}
